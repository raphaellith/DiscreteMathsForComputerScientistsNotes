\section{Lagrange's theorem and more on the properties of \(G^\times_m\)}

At the end of the previous section, we provided a complete walkthrough of the RSA algorithm using actual, concrete numbers. In that walkthrough, we showed that Bob's message can be decrypted by computing the value of \(13^{27}\) modulo 55.

This really goes without saying, but \(13^{27}\) is quite a big number, so how can computers perform 26 multiplications efficiently?

The answer is: they don't. In fact, there are a couple of tricks that we can use to speed up performing modular arithmetic on large integer powers.

In this section, we will illustrate these tricks with the following example problem:
%
\[7^{42} \text{ mod } 10 = \text{?}\]


\subsection{Repetitive squaring}

An obvious shortcut that we can take is repetitive squaring:
%
\begin{align*}
    7^2 &= 49\\
    7^4 &= 49^2 = 2401\\ 
    7^8 &= 2401^2 = 5764801\\
    7^{16} &= 5764801^2 = 33232930569601\\
    7^{32} &= 33232930569601^2 = 1104427674243920646305299201\\
    &\;\Downarrow\\
    7^{42} &= 7^{32} \times 7^8 \times 7^2 \\
    &= 1104427674243920646305299201 \times 5764801 \times 49\\
    &= 311973482284542371301330321821976049\\
    &= 9 \pmod{10}
\end{align*}
%
This way, we've successfully reduced the number of multiplications from 41 to 7; but can we do better?


\subsection{A better method}

Compared to repetitive squaring, a more elegant method  would be to make use of the fact that we can perform multiplication on the set of remainders.

We calculate the first few powers of \(7 \bmod{10}\), and discover that \(7^4 \bmod{10} = 1\).
%
\begin{alignat*}{3}
    7^1 &&&= 7 \tag{mod 10}\\
    7^2 &= 7 \times 7 = 49 &&= 9 \tag{mod 10}\\
    7^3 &= 9 \times 7 = 63 &&= 3 \tag{mod 10}\\
    7^4 &= 3 \times 7 = 21 &&= 1 \tag{mod 10}
\end{alignat*}
%
This allows us to simplify our expression as follows:
%
\begin{align*}
    7^{42} &= 7^{40} \times 7^2 \tag{mod 10}\\
    &= (7^4)^{10} \times 9 \tag{mod 10}\\
    &= 1^{10} \times 9 \tag{mod 10}\\
    &= 9\tag{mod 10}
\end{align*}
%
which results in the same answer we had before.

Despite its effectiveness, this method has the drawback that its very first step --- multiplying 7 until we get \(1 \text{ mod } 10\) --- involves a lot of trial and error, and we never know how long it will take us to finally hit 1.

To eliminate this issue, we will use a theorem known as Lagrange's theorem.


\subsection{Lagrange's theorem}

Lagrange's theorem states the following about groups and their (possibly but not necessarily cyclic) subgroups:
%
\begin{quote}
    \textbf{Lagrange's theorem}

    Any subgroup of a finite group must have an order that divides the order of the group.

    In other words, if we have a group of order \(n\) which contains a subgroup of order \(k\), then \(n\) must be divisible by \(k\).
\end{quote}
%
(The proof of this theorem is outside the scope of this course and therefore will not be included here.)

To illustrate this theorem, consider the group formed by the set of remainders \(G_6 = \{0, 1, 2, 3, 4, 5\}\) with respect to addition modulo \(6\). One of its subgroups consists of the set \(H = \{0, 2, 4\}\). Notice that \(\abs{H} = 3\) is a factor of \(\abs{G_6} = 6\), so Lagrange's theorem checks out.


\subsection{Applying Lagrange's theorem to cyclic subgroups of \(G^\times_m\)}

But how does Lagrange's theorem help us compute \(7^{42} \bmod{10}\)?

Let us generalize the problem to computing \(a^n \bmod{m}\). We will also throw in the assumption that \(a\) is coprime with \(m\).

To do this, we consider the multiplicative group of integers of modulo \(m\), i.e. \((G^\times_m, \times)\). Since we assumed that \(a\) and \(m\) are coprime, we know that \(a \in G^\times_m\). We can then use \(a\) to generate a cyclic subgroup \((H, \times)\) as follows:
%
\[H = \{1, a, a^2, \cdots, a^{k-1}\}\]
%
where \(k\) is the order of the element \(a\). Note that:
%
\begin{itemize}
    \item By definition of the order of an element (and of a cyclic subgroup), we have \(a^k = 1\).
    \item The subgroup \((H, \times)\) is of order \(k\).
    \item The group \((G^\times_m, \times)\) is of order \(\phi(m)\).
\end{itemize}
%
Combining these three facts with Lagrange's theorem, we have:
%
\begin{align*}
    a^{\phi(m)} &= a^{\abs{G^\times_m}}\\
    &= a^{l \cdot \abs{H}} \tag{for some integer \(l\); by Lagrange's theorem}\\
    &= a^{kl}\\
    &= (a^{k})^l\\
    &= 1^l\\
    &= 1
\end{align*}
%
This gives us the following result.
%
\begin{quote}
    If \(\gcd{(a, m)} = 1\), then \(a^{\phi(m)} = a^{\abs{G^\times_m}} = 1 \pmod{m}\).
\end{quote}
%
We can then use this result to calculate \(a^n \bmod{m}\). For instance, for \(7^{42} \bmod{10}\), we first note that 7 and 10 are coprime. Since \(\phi(10) = \abs{G^\times_{10}} = 4\), we have \(7^4 = 1 \pmod{m}\). Hence,
%
\begin{align*}
    7^{42} &= 7^{40} \times 7^2 \tag{mod 10}\\
    &= (7^4)^{10} \times 9 \tag{mod 10}\\
    &= 1^{10} \times 9 \tag{mod 10}\\
    &= 9\text{.} \tag{mod 10}
\end{align*}
%
This is similar to the method we demonstrated a few subsections ago, except that the trial and error procedure is no longer needed.