\section{Permutations}

\subsection{What is a permutation?}

A \textit{permutation} is an order in which a set of elements can be arranged. A permutation involving \(n\) elements is said to have \textit{degree} \(n\). In general, as we know from combinatorics, there are \(n!\) permutations of degree \(n\).

For instance, the integers 1, 2 and 3 can be arranged in 6 different ways, each of which is considered a different permutation of degree 3.
%
\begin{center}
    (1, 2, 3)\\
    (1, 3, 2)\\
    (2, 1, 3)\\
    (2, 3, 1)\\
    (3, 1, 2)\\
    (3, 2, 1)
\end{center}
%
We can denote each of these permutations using the following two-line notation.
%
\begin{alignat*}{5}
&
\begin{pmatrix}
    1 & 2 & 3\\
    1 & 2 & 3
\end{pmatrix}
&&\;\;&&
\begin{pmatrix}
    1 & 2 & 3\\
    1 & 3 & 2
\end{pmatrix}
&&\;\;&&
\begin{pmatrix}
    1 & 2 & 3\\
    2 & 1 & 3
\end{pmatrix}
\\&&&&&&&&&\\
&
\begin{pmatrix}
    1 & 2 & 3\\
    2 & 3 & 1
\end{pmatrix}
&&&&
\begin{pmatrix}
    1 & 2 & 3\\
    3 & 1 & 2
\end{pmatrix}
&&&&
\begin{pmatrix}
    1 & 2 & 3\\
    3 & 2 & 1
\end{pmatrix}
\end{alignat*}


\subsection{Viewing permutations as bijections}

Looking at the two-line notation, one might notice that any given permutation is merely a bijective function in disguise. For example, the permutation 
%
\[
\begin{pmatrix}
    1 & 2 & 3\\
    3 & 1 & 2
\end{pmatrix}
\]
%
can be represented by a bijective function \(\sigma : \{1, 2, 3\} \rightarrow \{1, 2, 3\}\), where
%
\[
\begin{cases}
    \sigma(1) = 3\\
    \sigma(2) = 1\\
    \sigma(3) = 2\text{.}
\end{cases}
\]
%
In fact, we can define a permutation of degree \(n\) to be a bijection \(\sigma : \{1, 2, 3, \cdots, n\} \rightarrow \{1, 2, 3, \cdots, n\}\). We also define the \textit{symmetric group} of degree \(n\), denoted as \(S_n\), to be the set of all permutations of degree \(n\).

For example, the symmetric group of degree 3 has the following elements.
%
\[
S_3 = \left\{
\begin{pmatrix}
    1 & 2 & 3\\
    1 & 2 & 3
\end{pmatrix},
%
\begin{pmatrix}
    1 & 2 & 3\\
    1 & 3 & 2
\end{pmatrix},
%
\begin{pmatrix}
    1 & 2 & 3\\
    2 & 1 & 3
\end{pmatrix},
%
\begin{pmatrix}
    1 & 2 & 3\\
    2 & 3 & 1
\end{pmatrix},
%
\begin{pmatrix}
    1 & 2 & 3\\
    3 & 1 & 2
\end{pmatrix},
%
\begin{pmatrix}
    1 & 2 & 3\\
    3 & 2 & 1
\end{pmatrix}
\right\}
\]

Now, consider the permutation
%
\[\epsilon = \begin{pmatrix}
    1 & 2 & 3\\
    1 & 2 & 3
\end{pmatrix}\text{.}\]
%
As the order of the elements is unchanged, this bijection is simply the identity function. Such a permutation, where \(\forall k \in \{1, 2, 3, \cdots, n\}, \epsilon(k) = k\), is called an \textit{identity permutation}.


\subsection{Composition and order of permutations}

Viewing permutations as bijections allows us to composite permutations in the same way that we composite functions. An example of this is displayed below.

\begin{figure}[H]
    \centering
    \begin{tikzpicture}    
        \node[align=center, anchor=south] at (-0.5, 0) {\(
            \sigma =
            \begin{pmatrix}
                1 & 2 & 3\\
                {\color{MidnightBlue} 3} & {\color{MidnightBlue} 1} &{\color{MidnightBlue} 2}
            \end{pmatrix}
        \)};

        \draw[very thick, -latex, BrickRed] (0, 0) -- (0, -1) node[pos=0.5, anchor=west]{\(\sigma\)};

        \node[align=center, anchor=north] at (0, -1) {\(
            \begin{pmatrix}
                {\color{MidnightBlue} 3} & {\color{MidnightBlue} 1} &{\color{MidnightBlue} 2}\\

                {\color{Goldenrod4} 2} & {\color{Goldenrod4} 3} &{\color{Goldenrod4} 1}
            \end{pmatrix}
        \)};

        \draw[very thick, -latex, BrickRed] (0, -2.5) -- (0, -3.5) node[pos=0.5, anchor=west]{\(\sigma\)};

        \node[align=center, anchor=north] at (0, -3.5) {\(
            \begin{pmatrix}
                {\color{Goldenrod4} 2} & {\color{Goldenrod4} 3} &{\color{Goldenrod4} 1}\\

                {\color{ForestGreen} 1} & {\color{ForestGreen} 2} &{\color{ForestGreen} 3}
            \end{pmatrix}
        \)};

        \draw[pen colour={BrickRed}, decoration={calligraphic brace, amplitude=10pt, raise=30pt},decorate, ultra thick] (1,1.2) -- (1,-4.8) node[pos=0.5, anchor=west, BrickRed, align=left, shift={(1.5, 0)}] {Applying \(\sigma\) three times (i.e. \(\sigma^3\))\\ gives us the identity permutation \(\epsilon\),\\ so \(\sigma^3 = \epsilon\)};
    \end{tikzpicture}
    \caption{An example of compositing permutations.}
    \label{fig:ch4-permutation-composition}
\end{figure}


We define the \textit{order} of a permutation \(\sigma\) to be the smallest positive integer \(k\) such that \(\sigma^k = \epsilon\), where \(\epsilon\) is the identity permutation. For example, in figure \ref{fig:ch4-permutation-composition}, applying \(\sigma\) three times takes us back to the start. This means that \(\sigma^3 = \epsilon\), the order of \(\sigma\) is 3.


\subsection{Disorders, and the sign of a permutation}\label{sec:sign-of-permutation}

For any given permutation \(\sigma\), a \textit{disorder} is a pair of values \((x, y)\) where \(x < y\) but \(\sigma(x) > \sigma(y)\).

If a permutation has an even number of disorders, then it is said to have a sign of \(+1\). If the number of disorders is odd, then the permutation has a sign of \(-1\).

For example, the permutation
%
\[
\sigma =
\begin{pmatrix}
    1 & 2 & 3\\
    3 & 1 & 2
\end{pmatrix}
\text{,}
\]
%
has two disorders: \((3, 1)\) and \((3, 2)\). Therefore, its sign must be \(\sgn(\sigma) = +1\).

Note that the sign of a permutation is defined in such a way that for any two permutations \(\sigma_1\) and \(\sigma_2\), we have
%
\[\sgn(\sigma_1 \sigma_2) = \sgn(\sigma_1) \cdot \sgn(\sigma_2)\text{.}\]
