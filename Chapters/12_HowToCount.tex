\section{How to count}

This section presents two counting principles which are fundamental to combinatorics.

Firstly, the \textit{multiplication principle} is used to count the number of tuples \((t_1, t_2, \cdots, t_n)\) where \(t_i\) are selected from independent sources. For any sets \(A_1, A_2, \cdots, A_n\), the cardinality of their Cartesian product is given by
%
\[
\abs{ A_1 \times A_2 \times \cdots \times A_n } = \abs{A_1} \times \abs{A_2} \times \cdots \times \abs{A_n}\text{.}
\]

Secondly, for any pair of sets \(A\) and \(B\), we have
%
\[\abs{A \cup B} = \abs{A} + \abs{B} - \abs{A \cap B}\text{.}\]
%
The same principle can be applied to calculating probabilities --- for any events \(A\) and \(B\), we have
%
\[P(A \lor B) = P(A) + P(B) - P(A \land B) \text{.}\]