\section{Binary relations}

\subsection{What is a binary relation?}

A \textit{binary relation} \(R(x, y)\) describes some relationship between \(x\) and \(y\) where \(x \in X\), \(y \in Y\) and \(R \subseteq X \times Y\). Examples include:
%
\begin{itemize}
    \item ``\(x < y\)''.
    \item ``\(y = x^2\)''.
    \item ``\(x\) is a child of \(y\)''.
    \item ``\(x\) and \(y\) are students from the same school''.
\end{itemize}


\subsection{Equivalence relations}

A binary relation is said to be an \textit{equivalence relation} on the set \(X\) if and only if it satisfies all of the following conditions.
%
\begin{enumerate}
    \item \textbf{Reflexivity:} Any element in \(X\) must be considered equivalent to itself.
    \[\forall x \in X,\; E(x, x)\]

    \item \textbf{Symmetry:} If \(x\) is considered equivalent to \(y\), then \(y\) must also be considered equivalent to \(x\).
    \[\forall x, y \in X,\; E(x, y) \Leftrightarrow E(y, x)\]

    \item \textbf{Transitivity:} If \(x\) is considered equivalent to \(y\), and \(y\) is considered equivalent to \(z\), then it follows that \(x\) must be considered equivalent to \(z\).
    \[\forall x, y, z \in X,\; E(x, y) \wedge E(y, z) \Rightarrow E(x, z)\]
\end{enumerate}

Examples of equivalence relations include
\begin{itemize}
    \item ``is equal to'' on the set of numbers;
    \item ``is similar to'' on the set of all triangles; and
    \item ``is congruent to, mod \(m\)'' on the set of integers.
\end{itemize}


\subsection{Equivalence classes}

Suppose we have an equivalence relation \(E(x, y)\) acting on a set \(X\). For any element \(a \in X\), we define its \textit{equivalence class} \([a]\) as the set of all elements in \(X\) that are considered ``equivalent'' to \(a\):
%
\[[a] := \{x \in X \setvert E(x, a)\}\]
%
Note that by definition \([a]\) is a subset of \(X\).

For example, consider the equivalence relation ``\(x - y\) is even'' which acts on the set of all integers. This gives us the following equivalence classes for 0 and 1.
%
\begin{align*}
    [0] &= \{0, 2, -2, 4, -4, \cdots, 2n, -2n, \cdots\}\\
    [1] &= \{1, -1, 3, -3, \cdots, 2n+1, -2n-1, \cdots\}
\end{align*}

